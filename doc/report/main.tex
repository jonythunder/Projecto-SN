\documentclass{article}

% ------------------------------ 76 characters -----------------------------
\usepackage[l2tabu, orthodox]{nag}

% -- Texto e codificação
\usepackage[utf8]{inputenc}
\usepackage[portuguese]{babel}
\usepackage{microtype}
\usepackage{xcolor}

% -- Tipo de letra
\usepackage{heuristica}
\usepackage[heuristica,vvarbb,bigdelims]{newtxmath}
\usepackage[T1]{fontenc}
%\renewcommand*\oldstylenums[1]{\textosf{#1}}

%\usepackage{lmodern}
%\usepackage[T1]{fontenc}

% -- Definir margens do documento
\usepackage{geometry}
\geometry{verbose, nomarginpar,
    tmargin = 2.5cm,
    bmargin = 2.5cm,
    lmargin = 2.5cm,
    rmargin = 2.5cm}
%\usepackage{showframe}

% -- Cabeçalho e rodapé
\usepackage{fancyhdr}
\fancyhf{}
\fancyhead[L]{\textsc{Header}}
\fancyhead[R]{\textsc{Footer}}
\fancyfoot[C]{\thepage}
\pagestyle{fancy}

% -- Funções matemáticas extra
\usepackage{mathtools}
\usepackage{siunitx}

% -- Símbolos extra
\usepackage{amssymb}
\usepackage{textcomp}
\usepackage{gensymb}
\usepackage{cancel}

% -- Referências
\usepackage{hyperref}

% --  Definições de imagens
\usepackage{graphicx}
\graphicspath{{graphics/}}
\usepackage{caption}
\usepackage{subcaption}

% -- Desenhar circuitos elétricos e lógicos
\usepackage{tikz}
\usepackage{pgfplots}
\usetikzlibrary{arrows.meta,positioning}
\pgfplotsset{compat=1.5}
\pgfplotsset{table/search path = {data}}
\pgfplotsset{	/pgf/number format/use comma,}

\begin{document}

\thispagestyle{empty}

\includegraphics[viewport=9.5cm 11cm 0cm 0cm,scale=0.29]{IST_A_CMYK_POS}
	
\begin{center}
	\vspace{70mm} % --  Espaço em branco
	\rule{\linewidth}{0.5pt} \\
    \vspace{2mm}
	\Huge \textsc{Projecto de Sistemas de Navegação} \\
	\rule{\linewidth}{2pt} \\
	\vspace{8mm} % -- Espaço em branco
	\LARGE Desenvolvimento de um sistema de detecção de intromissão em áreas restritas baseado em GPS diferencial
	
	\vspace{\fill} % --  Espaço em branco variável
	\large
	
	\begin{tabular}{r l}
		Tiago \textsc{Matias} & \textbf{65655} \\
		João \textsc{Manito} & \textbf{73096} \\
		Daniel \textsc{de Schiffart} & \textbf{81479}
	\end{tabular}
	
	\vspace{10mm} % --  Espaço em branco
	\Large Instituto Superior Técnico \\
	Mestrado em Engenharia Aeroespacial \\
	\vspace{1mm}
	\large Gestão de Projetos
	
	\vspace{10mm} % --  Espaço em branco
	\Large Junho de 2018
\end{center}

\newpage

\section{Objectivo}

\begin{itemize}
	\item Formular matematicamente a solução de GPS diferencial (ir buscar a literatura/internet)
	\item Preparar um algoritmo que permita definir um volume para a área restrita (em coordenadas ECEF(XYZ)) 
	\item Ter como opção de output um ficheiro KML desse volume
	\item Preparar um algoritmo que compute as coordenadas do receptor no referencial ECEF e determine intrusão na área restrita
	\item Inicialmente não será feita análise em tempo real nem usando DGPS, de forma a poder testar mais rapidamente os resultados, usando para tal um receptor GPS de smartphone a fazer logging da posição e detectar quando/onde foram feitas intrusões
	\item Numa primeira fase usar apenas uma área restrita para validar o algoritmo, criando áreas restritas adicionais posteriormente (e voltando a validar)
	\item Criar um algoritmo que receba o output do receptor GPS usando porta série e descodifique o resultado (se necessário)
	\item Criar um algoritmo para obter a informação da ground-station (se necessário)
	\item Criar um algoritmo que use o output do receptor GPS e a informação da ground-station para determinar a posição usando DGPS
	\item Testar o algoritmo (em tempo real) de forma a determinar a intrusão em apenas uma área restrita e posteriormente em mais que uma, com output gráfico de aviso de intrusão
	\item Gerar um ficheiro KML com o percurso realizado e as áreas restritas, bem como tendo as intrusões devidamente identificadas
\end{itemize}

\subsection{Bonus objectives (a pensar nisto só depois de acabado o projecto)}
Obter a dinâmica do utilizador (velocidade, rumo) e originar aviso se, à velocidade actual, for entrar na zona restrita em menos de X segundos

\end{document}
